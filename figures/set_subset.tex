\documentclass[tikz,border=2mm]{standalone}
\usepackage{amsmath}
\usepackage{amssymb}
\usepackage{pgfplots}
\usetikzlibrary{arrows.meta}
\usetikzlibrary{positioning, shapes, calc}

\tikzset{set/.style={draw,circle,inner sep=0pt,align=center}}
\tikzset{line/.style = {draw,thick, shorten >=-2pt, shorten <=-2pt}}
\tikzset{tick/.style={draw, minimum width=0pt, minimum height=2pt, inner sep=0pt, label=below:$#1$},tick/.default={}}

\begin{document}
	\colorlet{circle edge}{black!60}
	\colorlet{circle area}{black!20}
	
	\tikzset{filled/.style={fill=circle area, draw=circle edge, thick},
		outline/.style={draw=circle edge, thick}}
	
	\setlength{\parskip}{5mm}
	
	% Set A and B with A being a subset of B
	\begin{tikzpicture}
		% Define the larger circle B
		\def\outercircle{(0,0) circle (2.5cm)}
		% Define the smaller circle A inside B
		\def\innercircle{(0,0) circle (1.5cm)}
		
		% Draw the outer circle B
		\draw[outline] \outercircle node {};
		% Draw the inner circle A
		\draw[outline] \innercircle node {};
		
		% Label for the sets
		\node at (0,1.8) {$A$};
		\node at (0,2.8) {$B$};
	\end{tikzpicture}
\end{document}
