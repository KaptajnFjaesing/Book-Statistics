\section{Framing of Statistics}
\label{sec:framing_statistics}
In this book, statistics is framed as a game against Nature, following conventions from decision theory\cite{lavalle2006planning}. In this game, there are two players, whose roles are formalized in \dfref{def:robot} and \dfref{def:nature}.

\begin{definition}[Robot]
	\label{def:robot}
	The Robot is the primary decision maker in the statistical game.\index{Decision maker Robot}
\end{definition}

\begin{definition}[Nature]
	\label{def:nature}	
	Nature is an unpredictable decision maker that can interfere with the Robot's outcomes. It models uncertainty in the decision-making process.\index{Decision maker Nature}
\end{definition}

\begin{remark}[Statistical game setup]
	The game between the Robot and Nature is formalized by a probability space $(\Omega, \mathcal{F}, \mathbb{P})$, a parameter space $\Omega_W$, and a set of probability distributions $\mathcal{P}$ parameterized by $w\in \Omega_W$. The Robot and Nature each make a decision by choosing actions $u \in \Omega_U$ and $s \in \Omega_S$, respectively. The Robot receives a penalty from a cost function depending on both actions.\index{Game against Nature}
\end{remark}

\begin{definition}[Cost Function]
	\label{def:cost_function}
	A cost function associates a numerical penalty depending on decision $u \in \Omega_U$ and $s \in \Omega_S$,
	\begin{equation}
		C: \Omega_U \times \Omega_S \to \mathbb{R}.
	\end{equation}
\end{definition}
Given the observation $X=x$ as well as a set of past observations and matching actions of Nature
\begin{equation}
	D=\{(X = x_i,S= s_i)\}_{i=1}^n,
\end{equation}
the Robot's objective is to formulate a decision rule that minimize the expected cost associated with its decisions\cite{murphy2023probabilistic}.
\begin{definition}[Decision Rule]
	\label{def:decision_rule}
	A decision rule is a function $U$ that prescribes an action based on the current observation and past data. Formally, let $x \in \Omega_X$ be a new observation and $D \in (\Omega_X \times \Omega_S)^n$ denote the past observations and corresponding actions of Nature. Then a decision rule is a mapping
	\begin{equation}
		U: \Omega_X \times (\Omega_X \times \Omega_S)^n \to \Omega_U,
	\end{equation}
	where $\Omega_U$ is the action space of the Robot.
\end{definition}

\newpage
\begin{example}
	\label{ex:rain}
	Suppose the Robot has an umbrella and considers if it should bring it on a trip outside, i.e.
	\begin{equation}
		\Omega_U = \{"\text{bring umbrella}", "\text{don't bring umbrella}"\}.
	\end{equation}
	Nature have already picked whether or not it will rain later, i.e.
	\begin{equation}
		\Omega_S = \{"\text{rain}", "\text{no rain}"\},
	\end{equation}
	so the Robot's task is to estimate Nature's decision regarding rain later and either bring the umbrella or not. The Robot's decision rule, denoted as $U$, maps the available information $X=x$ (possibly $X=$ weather forecasts, current weather conditions, etc.) to one of its possible actions. For instance, $U(\text{weather forecast}, D)$ might map to the action "\text{bring umbrella}" if rain is predicted and "\text{don't bring umbrella}" otherwise.
\end{example}

The random variable\index{Random variable} $X: \Omega \to \Omega_X$ represent the information available (the information may be missing or null) to the Robot regarding the decision Nature will make, while $S: \Omega \to \Omega_S$ represent the different possible decisions of Nature. $\Omega_X$ and $\Omega_S$ have associated $\sigma$-algebras\index{$\sigma$-algebra} and probability measures\index{Probability measure}, however, such details are assumed to be understood in the practical application of statistics. 

\begin{remark}[Relaxation of Notation]
	\label{sec:notation}
	The formal measure-theoretic details introduced so far provide the foundation for statistical reasoning. In practice, however, many of these technicalities can be safely abstracted to facilitate computations and exposition.
	Accordingly, in the remainder of this book, the notation around probability spaces, $\sigma$-algebras, and probability measures will be \emph{relaxed}. Specifically:
	\begin{itemize}
		\item The symbol $p$ will be used informally to denote probability distributions, densities, mass functions, or measures.
		\item The probability of a random variable taking a specific value, $p(X=x)$, will usually be written as $p(x)$ for brevity.
	\end{itemize}
	
	This simplification allows advanced manipulation of probabilities without cumbersome formalism. Nevertheless, familiarity with the formal definitions provided here remains beneficial for a rigorous understanding.
\end{remark}

Given the observation $X=x$, as well as data $D$, the objective of the Robot is to minimize the expected cost associated with its decisions\cite{murphy2023probabilistic}
\begin{equation}
	\begin{split}
		\mathbb{E}[C(U, S)|I] &= \int dD dx ds  C(U(x,D),s) p(X=x,S=s,D|I)\\
		& = \int d\tilde{D} ds  C(U(\tilde{D}),s) p(S=s,\tilde{D}|I)
	\end{split}
	\label{eq:conditional_expected_cost}
\end{equation}
where $\tilde{D} = \{D,X= x\}$, $I$ denotes the background information (\dfref{def:background_information}) and the Robot aims to find the decision rule (\dfref{def:decision_rule}) which minimizes \EQref{eq:conditional_expected_cost}, meaning
\begin{equation}
	U^* = \arg\min_{U} \mathbb{E}[C(U, S)|I].
	\label{eq:decision_rule_x}
\end{equation}	
From \thref{theorem:total_expectation}
\begin{equation}
	\mathbb{E}[C(U, S)|I] = \mathbb{E}_{\tilde{D}}[\mathbb{E}_{S|\tilde{D}}[C(U, S)|\tilde{D},I]].
	\label{eq:total2}
\end{equation}
Using \EQref{eq:total2} in \EQref{eq:decision_rule_x}
\begin{equation}
	\begin{split}
		U^* &= \arg\min_{U} \mathbb{E}_{\tilde{D}}[\mathbb{E}_{S|\tilde{D}}[C(U, S)|\tilde{D},I]]\\
		&= \arg\min_{U} \int d\tilde{D}p(\tilde{D}|I) \mathbb{E}_{S|\tilde{D}}[C(U, S)|\tilde{D},I].
	\end{split}
	\label{eq:decision_rule2}
\end{equation}
Since $p(\tilde{D}|I)$ is a non-negative function, the minimizer of the integral is the same as the minimizer of the conditional expectation, meaning
\begin{equation}
	\begin{split}
		U^*(\tilde{D}) &= \arg\min_{U(\tilde{D})} \mathbb{E}_{S|\tilde{D}}[C(U(\tilde{D}), S)|\tilde{D},I]\\
		& = \arg\min_{U(\tilde{D})}\int  ds C(U(\tilde{D}),s) p(s|\tilde{D},I).
	\end{split}
	\label{eq:decision_rule3}
\end{equation}
\begin{example}
	In general the random variable $X$ represent the observations the Robot has available that are related to the decision Nature is going to make. However, this information may not be given, in which case $\{x,D_x\}=\emptyset$ and consequently
	\begin{equation}
		\begin{split}
			\tilde{D} &=\{S= s_i\}_{i=1}^n\\
			&\equiv D_s.
		\end{split}
	\end{equation}
	In this case, the Robot is forced to model the decisions of Nature with a probability distribution with associated parameters without observations. From \EQref{eq:decision_rule3} the optimal action for the Robot can be written
	\begin{equation}
		U^*(D_s) = \arg\min_{U(D_s)} \mathbb{E}_{S|\tilde{D}}[C(U(\tilde{D}), S)|\tilde{D},I]
		\label{eq:best_decision1}
	\end{equation}
\end{example}

\subsection{Assigning a Cost Function}
\label{sec:assing_cost}
The cost function (see definition \ref{def:cost_function}) associates a numerical penalty to the Robot's action and thus the details of it determine the decisions made by the Robot. Under certain conditions, a cost function can be shown to exist~\citep{lavalle2006planning}, however, there is no systematic way of producing or deriving the cost function beyond applied logic. In general, the topic can be split into considering a continuous and discrete action space, $\Omega_U$. 	

\subsubsection{Continuous Action Space}
In case of a continuous action space, the cost function is typically picked from a set of standard choices.	
\begin{definition}[Linear Cost Function]
	\label{def:linear_cost_function}
	The linear cost function is defined viz
	\begin{equation}
		C(U(\tilde{D}),s) \equiv |U(\tilde{D})-s|.
	\end{equation}
	
\end{definition}
\begin{theorem}[Median Decision Rule]
	\label{def:median_decision_rule}
	Assuming the cost function of \dfref{def:linear_cost_function}
	\begin{equation}
		\begin{split}
			\mathbb{E}_{S|\tilde{D}}[C(U(\tilde{D}), S)|\tilde{D},I] &= \int_{-\infty}^{\infty} ds |U(\tilde{D})-s| p(s|\tilde{D},I)\\
			&= \int_{-\infty}^{U(\tilde{D})}ds (s-U(\tilde{D}))p(s|\tilde{D},I)\\
			&\quad+\int_{U(\tilde{D})}^\infty ds (U(\tilde{D})-s)p(s|\tilde{D},I)\\
		\end{split}
	\end{equation}
	\begin{equation}
		\begin{split}
			0 &=\frac{\partial \mathbb{E}_{S|\tilde{D}}[C(U(\tilde{D}), S)|\tilde{D},I]}{\partial U(\tilde{D})}\bigg|_{U(\tilde{D})=U^*(\tilde{D})}\\
			&= (U^*(\tilde{D})-U^*(\tilde{D}))p(U^*(\tilde{D})|\tilde{D},I)+\int_{-\infty}^{U^*(\tilde{D})}ds p(s|\tilde{D},I)\\
			&\quad+(U^*(\tilde{D})-U^*(\tilde{D}))p(U^*(\tilde{D})|\tilde{D},I)-\int_{U^*(\tilde{D})}^\infty ds p(s|\tilde{D},I)
		\end{split}
	\end{equation}
	\begin{equation}
		\begin{split}
			\int_{-\infty}^{U^*(\tilde{D})}ds p(s|\tilde{D},I) &= \int_{U^*(\tilde{D})}^\infty ds p(s|\tilde{D},I)\\
			&= 1- \int_{-\infty}^{U^*(\tilde{D})} ds p(s|\tilde{D},I)\\
		\end{split}
	\end{equation}
	\begin{equation}
		\int_{-\infty}^{U^*(\tilde{D})} ds p(s|\tilde{D},I) = \frac{1}{2}
	\end{equation}
	which is the definition of the median.
\end{theorem}

\begin{definition}[Quadratic Cost Function]
	\label{def:quadratic_cost}
	The quadratic cost function is defined as
	\begin{equation}
		C(U(\tilde{D}),s) \equiv (U(\tilde{D})-s)^2.
	\end{equation}
\end{definition}

\begin{theorem}[Expectation Decision Rule]
	\label{theorem:expectation_decision_rule}
	Assuming the cost function of \dfref{def:quadratic_cost}
	\begin{equation}
		\begin{split}
			\mathbb{E}_{S|\tilde{D}}[C(U(\tilde{D}), S)|\tilde{D},I] &= \int ds (U(\tilde{D})-s)^2 p(s|\tilde{D},I)\\
			&\Downarrow\\
			\frac{\partial \mathbb{E}_{S|\tilde{D}}[C(U(\tilde{D}), S)|\tilde{D},I]}{\partial U(\tilde{D})}\bigg|_{U(\tilde{D})=U^*(x)} &= 2U^*(\tilde{D})-2\int ds sp(s|\tilde{D},I)\\
			&=0\\
			&\Downarrow\\
			U^*(\tilde{D})& = \int ds sp(s|\tilde{D},I)\\
			&= \mathbb{E}_{S|\tilde{D}}[S|\tilde{D},I]
		\end{split}
	\end{equation}
	which is the definition of the expectation value.
\end{theorem}

\begin{definition}[0-1 Cost Function]
	\label{def:0_1_cost_function}
	The 0-1 cost function is defined viz
	\begin{equation}
		C(U(\tilde{D}),s) \equiv 1-\delta(U(\tilde{D})-s).
	\end{equation}
\end{definition}

\begin{theorem}[MAP Decision Rule]
	\label{theorem:MAP}
	The maximum aposteriori (MAP) follows from assuming 0-1 loss viz
	\begin{equation}
		\mathbb{E}_{S|\tilde{D}}[C((\tilde{D}), S)|\tilde{D},I] = 1-\int ds \delta(U(\tilde{D})-s) p(S = s|\tilde{D},I)
	\end{equation}
	meaning
	\begin{equation}
		\begin{split}
			\frac{\partial \mathbb{E}_{S|\tilde{D}}[C(U(\tilde{D}), S)|\tilde{D},I]}{\partial U(\tilde{D})}\bigg|_{U(\tilde{D})=U^*(\tilde{D})} &= -\frac{\partial p(S = U(\tilde{D})|\tilde{D},I)}{\partial U(\tilde{D})}\bigg|_{U(\tilde{D})=U^*(\tilde{D})}\\
			&=0\\
		\end{split}
	\end{equation}
	which is the definition of the MAP.
\end{theorem}


\begin{example}
	The median decision rule is symmetric with respect to $z(\tilde{D},s) \equiv U(\tilde{D})-s$, meaning underestimation ($z<0$) and overestimation ($z>0$) is penalized equally. This decision rule can be generalized to favoring either scenario by adopting the cost function
	\begin{equation}
		C(U(\tilde{D}), s) = \alpha\cdot \operatorname{swish}(U(\tilde{D})-s,\beta)
		+(1-\alpha)\cdot \operatorname{swish}(s-U(\tilde{D}),\beta),
	\end{equation}
	where
	\begin{equation}
		\operatorname{swish}(z,\beta) = \frac{z}{1+e^{-\beta z}}.
	\end{equation}
	Taking $\alpha \ll 1$ means $z<0$ will be penalized relatively more than $z>0$. The expected cost is
	\begin{equation}
		\mathbb{E}_{S|\tilde{D}}[C(U(\tilde{D}), S)|\tilde{D},I] = \int ds  p(S=s|\tilde{D},I)\, C(U(\tilde{D}),s).
	\end{equation}
	The derivative of the expected cost with respect to the decision rule can be approximated viz
	\begin{equation}
		\begin{split}
			\frac{dC}{dU} & = \frac{dC}{dz}\frac{dz}{dU}\\
			& = \bigg(\frac{\alpha}{1+e^{-\beta z}}-\frac{1-\alpha}{1+e^{\beta z}}\\
			&\qquad+\frac{\alpha\beta e^{-\beta z}z}{(1+e^{-\beta z})^2}+\frac{(1-\alpha)\beta e^{\beta z}z}{(1+e^{\beta z})^2}\bigg)\frac{dz}{dU}\\
			&= \frac{\beta z e^{\beta z}-e^{\beta z}-1}{(1+e^{\beta z})^2}+\alpha+\mathcal{O}(\alpha^2)\\
			&\approx  \alpha -\frac{1}{(1+e^{\beta z})^2}
		\end{split}
	\end{equation}
	leading to the approximate expected cost
	\begin{equation}
		\begin{split}
			\frac{d\mathbb{E}_{S|\tilde{D}}[C(U(\tilde{D}), S)|\tilde{D},I]}{dU(\tilde{D})} &\approx \int ds p(s|\tilde{D},I) \bigg(\alpha -\frac{1}{(1+e^{\beta z(\tilde{D},s)})^2}\bigg)\\
			& = \alpha -\int ds p(s|\tilde{D},I)\frac{1}{(1+e^{\beta z(\tilde{D},s)})^2}.\\
			& = 0
		\end{split}
	\end{equation}
	For large $\beta$, the factor $\frac{1}{(1+e^{\beta (U(\tilde{D})-s)})^2}$ approaches the indicator $\mathbb{1}\{s>U(\tilde{D})\}$. Hence,
	\begin{equation}
		\int_{-\infty}^{\infty} ds p(s|\tilde{D},I)\frac{1}{(1+e^{\beta z(\tilde{D},s)})^2} \approx \int_{U(\tilde{D})}^{\infty} ds p(s|\tilde{D},I)
	\end{equation}
	This means the optimal decision rule can be written viz
	\begin{equation}
		\alpha \approx \int_{U(\tilde{D})}^{\infty} ds p(s|\tilde{D},I).
		\label{eq:quantile_decision_rule}
	\end{equation}
	The optimal decision $U^*(\tilde{D})$ is the $\alpha$-quantile of the conditional distribution $p(S|\tilde{D},I)$. This rule is known as the quantile decision rule.
\end{example}

\subsubsection{Discrete Action Space}
In case of a continuous action space, the conditional expected loss can be written
\begin{equation}
	\mathbb{E}_{S|\tilde{D}}[C(U(\tilde{D}), S)|\tilde{D},I] = \sum_{s\in \Omega_S}C(U(\tilde{D}),s)p(s|\tilde{D},I),
	\label{eq:conditional_cost_discrete}
\end{equation}
where the cost function is typically represented in matrix form viz
\begin{center}
	\begin{tabular}{ c  c  c  c  c  }
		&& $S$& & \\
		&& $s^{(1)}$ & \dots & $s^{(\text{dim}(\Omega_S))}$ \\
		\cline{3-5}
		$U(\tilde{D})$ & $u^{(1)}$& \multicolumn{1}{|l}{$C(u^{(1)}, s^{(1)})$} &\multicolumn{1}{l}{\dots}&\multicolumn{1}{l|}{$C(u^{(1)}, s^{(\text{dim}(\Omega_S))})$} \\
		& \vdots & \multicolumn{1}{|l}{\vdots} &\multicolumn{1}{l}{\vdots}&\multicolumn{1}{l|}{\vdots} \\
		& $u^{(\text{dim}(\Omega_U))}$ & \multicolumn{1}{|l}{$C(u^{(\text{dim}(\Omega_U))}, s^{(1)})$} &\multicolumn{1}{l}{\dots}&\multicolumn{1}{l|}{$C(u^{(\text{dim}(\Omega_U))}, s^{(\text{dim}(\Omega_S)}))$} \\
		\cline{3-5}
	\end{tabular}
\end{center}
Note that the upper index represent realized values of $s$ whereas a lower index represent datapoints.

\begin{example}
	With reference to \exref{ex:rain}, the possible states of Nature are $s^{(1)} = "\text{rain}"$ and $s^{(2)} = "\text{no rain}"$, whereas each observed outcome $s_i$ in the dataset 
	\begin{equation}
		D = \{(x_1,s_1),(x_2,s_2),(x_3,s_3)\}
	\end{equation}
	takes a value in $\{s^{(1)},s^{(2)}\}$. For instance, one possible dataset realization could be $s_1=s^{(1)}$, $s_2=s^{(1)}$, and $s_3=s^{(2)}$.
\end{example}

\begin{example}
	Consider a binary classification problem with action space $\Omega_U = \{u^{(1)},u^{(2)}\}$ and Nature's state space $\Omega_S = \{s^{(1)}, s^{(2)}\}$, where $u^{(1)}$ corresponds to predicting class $s^{(1)}$ and $u^{(2)}$ to predicting class $s^{(2)}$. Let
	\begin{equation}
		D = \{(x_i,s_i)\}_{i=1}^n
	\end{equation}
	denote the training data, where $s_i \in \Omega_S$ are observed realizations of Nature's states. Let $U(x,D)$ be a classifier based on the probability $p(S = s | x, D, I)$. Define a threshold $k\in[0,1]$ and the decision rule
	\begin{equation}
		U_k(x,D) =
		\begin{cases}
			u^{(1)}, & p(S=s^{(2)} | x, D, I) < k,\\
			u^{(2)}, & p(S=s^{(2)} | x, D, I) \ge k.
		\end{cases}
	\end{equation}
	For a fixed threshold $k$, classifier performance is summarized in the confusion matrix
	\begin{center}
		\begin{tabular}{ c  c  c c}
			&& $S$ &  \\
			&& $s^{(1)}$ & $s^{(2)}$ \\
			\cline{3-4}
			$U(x,D)$ & $u^{(1)}$& \multicolumn{1}{|l}{TP(k)} & \multicolumn{1}{l|}{FP(k)}\\
			& $u^{(2)}$& \multicolumn{1}{|l}{FN(k)} & \multicolumn{1}{l|}{TN(k)}\\
			\cline{3-4}
		\end{tabular}
	\end{center}
	and standard performance measures are defined as
	\begin{align}
		\operatorname{TPR(k)} &= \frac{\operatorname{TP(k)}}{\operatorname{TP(k)} + \operatorname{FN(k)}},\\
		\operatorname{FPR(k)} &= \frac{\operatorname{FP(k)}}{\operatorname{FP(k)} + \operatorname{TN(k)}},\\
		\operatorname{Accuracy(k)} &= \frac{\operatorname{TP(k)} + \operatorname{TN(k)}}{\operatorname{TP(k)} + \operatorname{TN(k)} + \operatorname{FP(k)} + \operatorname{FN(k)}}.
	\end{align}
	Varying the threshold $k$ over $[0,1]$ defines a family of classifiers $U_k(x,D)$, which induces a set of points
	\begin{equation}
		\operatorname{ROC} = \{ (\operatorname{FPR(k)}, \operatorname{TPR(k)}) : k \in [0,1] \}.
	\end{equation}
	The Area Under the ROC\index{AUROC}\index{Area Under the ROC} Curve (AUROC) is a threshold-independent measure. Let $X_{(s^{(1)})}$ and $X_{(s^{(2)})}$ denote independent draws from the class-conditional distributions $p(x | S = s^{(1)})$ and $p(x | S = s^{(2)})$, respectively. Then
	\begin{equation}
		\operatorname{AUROC} = p( p(S=s^{(2)} | X_{(s^{(2)})}, D, I) > p(S=s^{(2)} | X_{(s^{(1)})}, D, I)| D, I ),
	\end{equation}
	i.e., the probability that the classifier assigns a higher score to a randomly chosen positive instance than to a randomly chosen negative instance. Equivalently,
	\begin{equation}
		\operatorname{AUROC} = \int_0^1 \operatorname{TPR}(\operatorname{FPR}^{-1}(u)) \, du,
	\end{equation}
	under regularity conditions ensuring $\operatorname{FPR}$ is invertible. The Accuracy Ratio\index{Accuracy Ratio}\index{AR} (AR), or normalized Gini coefficient\index{Normalized Gini coefficient}, is defined from the AUROC as
	\begin{equation}
		\operatorname{AR} = 2 \cdot \operatorname{AUROC} - 1.
	\end{equation}
	and provide a measure rescaled to the interval $[-1,1]$.
	
\end{example}


\begin{example}
	Consider a discrete action space with an observation $X=x$ and available data $D$ ($\tilde{D} \equiv {x, D}$). Picking a class corresponds to an action, so classification can be viewed as a game against nature, where nature has picked the true class and the robot has to pick a class as well. Suppose there are only two classes and the cost function is defined by the matrix
	\begin{center}
		\begin{tabular}{ c  c  c  c }
			&& $S$& \\
			&& $s^{(1)}$ & $s^{(2)}$  \\
			\cline{3-4}
			$U(\tilde{D})$ & $u^{(1)}$& \multicolumn{1}{|l}{$0$} &\multicolumn{1}{l|}{$\lambda_{12}$}  \\
			& $u^{(2)}$& \multicolumn{1}{|l}{$\lambda_{21}$} & \multicolumn{1}{l|}{$0$} \\
			\cline{3-4}
		\end{tabular}
	\end{center}
	\begin{enumerate}
		\item Show that the decision $u$ that minimizes the expected loss is equivalent to setting a probability threshold $k$ and predicting $U(\tilde{D})=u^{(1)}$ if $p(S=s^{(2)}|\tilde{D},I) < k$ and $U(\tilde{D})=u^{(2)}$ if $p(S=s^{(2)}|\tilde{D},I)\geq k$. What is $k$ as a function of $\lambda_{12}$ and $\lambda_{21}$?\newline
		
		The conditional expected cost (\EQref{eq:conditional_cost_discrete})
		\begin{equation}
			\begin{split}
				\mathbb{E}_{S|\tilde{D}}[C(U(\tilde{D}), S)|\tilde{D},I] & = \sum_sC(U(\tilde{D}),S=s)p(S=s|\tilde{D},I)\\
				& = C(U(\tilde{D}),S=s^{(1)})p(S=s^{(1)}|\tilde{D},I)\\
				& \quad+C(U(\tilde{D}),S=s^{(2)})p(S=s^{(2)}|\tilde{D},I)\\
			\end{split}
		\end{equation}
		For the different possible actions
		\begin{equation}
			\begin{split}
				\mathbb{E}_{S|\tilde{D}}[C(u^{(1)}, S)|\tilde{D},I] &= \lambda_{12}p(S=s^{(2)}|\tilde{D},I),\\
				\mathbb{E}_{S|\tilde{D}}[C(u^{(2)}, S)|\tilde{D},I] &= \lambda_{21}p(S=s^{(1)}|\tilde{D},I),\\
			\end{split}
		\end{equation}
		$U(\tilde{D})=u_1$ iff
		\begin{equation}
			\mathbb{E}_{S|\tilde{D}}[C(u^{(1)},S)|\tilde{D},I]<\mathbb{E}_{S|\tilde{D}}[C(u^{(1)},S)|\tilde{D},I])
		\end{equation}
		meaning
		\begin{equation}
			\begin{split}
				\lambda_{12}p(S=s^{(2)}|\tilde{D},I)&<\lambda_{21}p(S = s^{(1)}|\tilde{D},I)\\
				&=\lambda_{21}(1-p(S =s^{(2)}|\tilde{D},I))
			\end{split}
		\end{equation}
		meaning $U(\tilde{D}) = u_1$ iff
		\begin{equation}
			p(S=s^{(2)}|\tilde{D},I)<\frac{\lambda_{21}}{\lambda_{12}+\lambda_{21}}=k
		\end{equation}
		
		
		\item Show a loss matrix where the threshold is $0.1$.\newline
		
		$k = \frac{1}{21}=\frac{\lambda_{21}}{\lambda_{12}+\lambda_{21}} \Rightarrow \lambda_{12}=9\lambda_{21}$ yielding the loss matrix
		
		\begin{center}
			\begin{tabular}{ c  c  c  c }
				&& $S$& \\
				&& $s^{(1)}$ & $s^{(2)}$  \\
				\cline{3-4}
				$U(\tilde{D})$ & $u^{(1)}$& \multicolumn{1}{|l}{$0$} &\multicolumn{1}{l|}{$9\lambda_{21}$}  \\
				& $u^{(2)}$& \multicolumn{1}{|l}{$\lambda_{21}$} & \multicolumn{1}{l|}{0} \\
				\cline{3-4}
			\end{tabular}
		\end{center}
		
		You may set $\lambda_{21}=1$ since only the relative magnitude is important in relation to making a decision.
		
	\end{enumerate}
	
	
\end{example}


\begin{example}
	In many classification problems one has the option of assigning $x$ to class $k\in K$ or, if the robot is too uncertain, choosing a reject option. If the cost for rejection is less than the cost of falsely classifying the object, it may be the optimal action. Define the cost function as follows
	\begin{equation}
		C(U(\tilde{D}),s)=\begin{cases}
			0 & \text{if correct classification ($U(\tilde{D})=s$)}\\
			\lambda_r & \text{if reject option ($U(\tilde{D})=\operatorname{reject}$)}\\
			\lambda_s & \text{if wrong classification ($U(\tilde{D})\neq s$)}\\
		\end{cases}.
	\end{equation}
	
	\begin{enumerate}
		\item Show that the minimum cost is obtained if the robot decides on class $U(\tilde{D})$ if
		\begin{equation}
			p(S=U(\tilde{D})|\tilde{D},I)\geq p(S\neq U(\tilde{D})|\tilde{D},I)
		\end{equation}
		 and if 
		 \begin{equation}
		 	p(S=U(\tilde{D})|\tilde{D},I)\geq 1-\frac{\lambda_r}{\lambda_s}.
		 \end{equation}
		The conditional expected cost if the robot does not pick the reject option, meaning $U(\tilde{D})\in \Omega_U\setminus\operatorname{reject}$
		\begin{equation}
			\begin{split}
				\mathbb{E}_{S|\tilde{D}}[C(U(\tilde{D}), S)|\tilde{D},I] & = \sum_s C(U(\tilde{D}),S=s)p(S=s|\tilde{D},I)\\
				&= \sum_{s\neq U(\tilde{D})}\lambda_sp(S=s|\tilde{D},I)\\
				&= \lambda_s(1-p(S=U(\tilde{D})|\tilde{D},I))
			\end{split}
			\label{eq:cost1a}
		\end{equation}
		where for the second equality it has been used that the cost of a correct classification is $0$, so the case of $S=U(\tilde{D})$ does not enter the sum. For the third equality it has been used that summing over all but $S=U(\tilde{D})$ is equal to $1-p(S=U(\tilde{D})|\tilde{D},I)$. The larger $p(S=U(\tilde{D})|\tilde{D},I)$, the smaller loss (for $\lambda_s>0$), meaning the loss is minimized for the largest probability. The conditional expected loss if the robot picks the reject option
		\begin{equation}
			\begin{split}
				\mathbb{E}_{S|\tilde{D}}[C(\operatorname{reject}, S)|\tilde{D},I]&= \lambda_r\sum_sp(S=s|\tilde{D},I)\\
				&=\lambda_r.
			\end{split}
			\label{eq:cost2a}
		\end{equation}
		Equation \eqref{eq:cost1a} show picking $\arg\max_{U(\tilde{D})\in \Omega_U\setminus \operatorname{reject}} p(S=U(\tilde{D})|\tilde{D},I)$ is the best option among classes $U(\tilde{D})\neq \operatorname{reject}$. To be the best option overall, it also needs to have lower cost than the reject option. Using \EQref{eq:cost1a} and \EQref{eq:cost2a} yields
		\begin{equation}
			(1-p(S=U(\tilde{D})|\tilde{D},I))\lambda_s< \lambda_r
		\end{equation}
		meaning
		\begin{equation}
			p(S=U(\tilde{D})|\tilde{D},I)\geq 1-\frac{\lambda_r}{\lambda_s}.
		\end{equation}
		
		\item Describe qualitatively what happens as $\frac{\lambda_r}{\lambda_s}$ is increased from $0$ to $1$.\newline
		
		\begin{equation}
			\frac{\lambda_r}{\lambda_s}=0
		\end{equation}
		means rejection is rated as a successful classification -- i.e. no cost associated -- and this become the best option (rejection that is) unless
		\begin{equation}
			p(S=U(\tilde{D})|\tilde{D},I)=1,
		\end{equation}
		corresponding to knowing the correct class with absolute certainty. In other words; in this limit rejection is best unless the robot is certain of the correct class. 
		\begin{equation}
			\frac{\lambda_r}{\lambda_s}=1
		\end{equation} 
		means rejection is rated a misclassification -- i.e. $\lambda_r=\lambda_s$ -- and thus and "automatic cost". Hence, in this case rejection is never chosen. In between the limits, an interpolation of interpretations apply.
	\end{enumerate}
\end{example}

\begin{remark}[Connection to Statistical Paradigms]
	So far in this chapter, there has been no reference to statistical paradigms (Bayesian or Frequentist). This is because all preceding material is valid under both the Bayesian (\dfref{def:bayesian_statistics}) and Frequentist (\dfref{def:frequentist_statistics}) paradigms. The difference between the two becomes apparent when considering the parameters of Nature's model.
\end{remark}
