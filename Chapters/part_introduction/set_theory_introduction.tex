\chapter{Introduction to Set Theory}
\label{chp:set_theory}
Set theory is a fundamental branch of mathematical logic that provides a foundation for much of mathematics, including probability theory. At its core, set theory deals with the concept of a set, which is a collection of distinct objects or elements. In this introduction, the essential properties and operations of sets are explored in order to lay the groundwork for the axiomatic formation of probability theory and statistics.


\begin{definition}[Membership]
	In set theory, the membership relation between an object $o$ and a set $A$ is fundamental. $o \in A$ denotes that $o$ is an element or member of $A$.
\end{definition}

\begin{definition}[Set]
	A set is a collection of distinct objects, considered as an object in its own right. Sets are typically denoted using curly braces $\{\}$ and can be described in two primary ways:
	\begin{enumerate}
		\item By listing its elements separated by commas, e.g., $A = \{a_1, a_2, a_3\}$.
		\item By specifying a characterizing property of its elements, e.g., $A = \{x \mid x \text{ is a natural number}\}$.
	\end{enumerate}
	Sets can also be illustrated graphically, as shown in \figref{fig:generic_set}.
	\begin{figure}[H]
		\centering
		\includegraphics[width = 0.3\textwidth]{figures/generic_set.pdf}
		\caption{The graphical representation of a generic set $A$ with generic elements $o$.}
		\label{fig:generic_set}
	\end{figure}
\end{definition}

\begin{definition}[Subset]
	A set $A$ is called a subset of a set $B$, denoted $A \subseteq B$, if every element of $A$ is also an element of $B$. Formally, $A\subseteq B$ if $\forall x \in A, x \in B$. By this definition, a set is always a subset of itself.
	\begin{figure}[H]
		\centering
		\includegraphics[width = 0.5\textwidth]{figures/set_subset.pdf}
		\caption{The graphical representation of $A\subseteq B$.}
		\label{fig:set_subset}
	\end{figure}
\end{definition}

\begin{definition}[Proper Subset]
	A set $A$ is called a proper subset of a set $B$, denoted $A \subset B$, if $A \subseteq B$ and $A \neq B$. This means that $A$ is a subset of $B$ but $A$ is not equal to $B$; there is at least one element in $B$ that is not in $A$.
\end{definition}

\begin{example}
	Suppose $A = \{\twemoji{banana}, \twemoji{apple}, \twemoji{eggplant}\}$, then $\{\twemoji{banana}, \twemoji{apple}\}$ and $\{\twemoji{apple}\}$ are proper subsets of $A$, meaning $\{\twemoji{banana}, \twemoji{apple}\},\{\twemoji{apple}\}\subset  A$. $\{\twemoji{banana}, \twemoji{carrot}\}$, on the other hand, is not a subset of $A$, meaning $\{\twemoji{banana}, \twemoji{carrot}\}\not\subset  A$.
\end{example}
\begin{example}
	$\twemoji{banana}$, $\twemoji{apple}$, and $\twemoji{eggplant}$ are members (elements) of the set $\{\twemoji{banana}, \twemoji{apple}, \twemoji{eggplant}\}$, but are not subsets of it; and in turn, the subsets, such as $\{\twemoji{banana}\}$, are not members of the set $\{\twemoji{banana}, \twemoji{apple}, \twemoji{eggplant}\}$.
\end{example}

\begin{definition}[Empty Set]
	The empty set, denoted by $\emptyset$ or $\{\}$, is the set that contains no elements.
\end{definition}

\begin{definition}[Universal Set]
	The universal set, denoted by $\Omega$, is the set that contains all the objects or elements under consideration in a particular discussion or problem. It is the largest set in the context of a given study.
\end{definition}

\begin{definition}[Closure]
	A set $A$ is said to be \textit{closed} under a certain operation if, for every pair of elements $x$ and $y$ in $A$, the result of applying the operation to $x$ and $y$ is also in $A$.
\end{definition}

\begin{definition}[Union]
	The union of sets $A$ and $B$, denoted by $A \cup B$, is defined as the set containing all elements that are in $A$ or $B$ (or both). \figref{fig:set_union} provide a graphical representation of $A \cup B$.
	\begin{figure}[h]
		\centering
		\includegraphics[width = 0.6\textwidth]{figures/set_union.pdf}
		\caption{The figure show the union of sets $A$ and $B$. Each circle represent the sets and the colored region represent the result of the result of the binary operation.}
		\label{fig:set_union}
	\end{figure}
\end{definition}

\begin{definition}[Intersection]
	The intersection of sets $A$ and $B$, denoted by $A \cap B$, is defined as the set containing all elements that are common to both $A$ and $B$. \figref{fig:set_intersection} provide a graphical representation of $A \cap B$.
	\begin{figure}[H]
		\centering
		\includegraphics[width = 0.6\textwidth]{figures/set_intersection.pdf}
		\caption{The figure show the intersection of sets $A$ and $B$. Each circle represent the sets and the colored region represent the result of the result of the binary operation.}
		\label{fig:set_intersection}
	\end{figure}
\end{definition}

\begin{definition}[Disjoint]
	Two sets $A$ and $B$ are said to be disjoint if their intersection is the empty set, i.e., $A \cap B = \emptyset$. \figref{fig:set_disjoint} provide a graphical representation of $A \cap B=\emptyset$.
	\begin{figure}[H]
		\centering
		\includegraphics[width = 0.8\textwidth]{figures/set_disjoint.pdf}
		\caption{The figure show the case where the intersection of sets $A$ and $B$ is the empty set. Each circle represent the sets and the colored region represent the result of the result of the binary operation.}
		\label{fig:set_disjoint}
	\end{figure}
\end{definition}


\begin{definition}[Complementation]
	The complement of set $A$, denoted by $A^c$, is defined as the set containing all elements in the universal set $\Omega$ that are not in $A$. \figref{fig:set_complementation} provide a graphical representation of $(A \cap B)^c$.
	\begin{figure}[H]
		\centering
		\includegraphics[width = 0.6\textwidth]{figures/set_complementary.pdf}
		\caption{The figure show the complementary of the intersection of sets $A$ and $B$. Each circle represent the sets and the colored region represent the result of the result of the binary operation.}
		\label{fig:set_complementation}
	\end{figure}
\end{definition}

\begin{definition}[Difference]
	The difference between set $A$ and $B$, denoted by $A \setminus B = A\cap B^c$, is defined as the set containing all elements in $A$ that are not in $B$. \figref{fig:set_minus} provide a graphical representation of $A\setminus B$ and $B\setminus A$.
	\begin{figure}[H]
		\centering
		\includegraphics[width = 0.6\textwidth]{figures/set_minus.pdf}
		\includegraphics[width = 0.6\textwidth]{figures/set_minus2.pdf}
		\caption{(left) show $A$ minus $B$ and (right) show $B$ minus $A$. Each circle represent the sets and the colored region represent the result of the result of the binary operation.}
		\label{fig:set_minus}
	\end{figure}
\end{definition}

\begin{definition}[Power Set]
	The power set of a set $A$, denoted by $2^A$, is defined as the set containing all possible subsets of $A$, including $A$ itself and the empty set.
\end{definition}

\begin{example}
	Suppose $A = \{a_1,a_2,a_3\}$, then
	\begin{equation}
		\begin{split}
			2^A = \{&\emptyset, \{a_1\}, \{a_2\}, \{a_3\}, \{a_1, a_2\},\\
			& \{a_1, a_3\}, \{a_2, a_3\}, \{a_1, a_2, a_3\}\}.
		\end{split}
	\end{equation}
\end{example}

\begin{definition}[Symmetric Difference]
	The symmetric difference of sets $A$ and $B$, denoted by $A \Delta B$, is defined as the set containing all elements that are in either $A$ or $B$ but not in both, meaning $A \Delta B = (A \cap B)^c$. \figref{fig:set_complementation} show the symmetric difference between sets $A$ and $B$.
\end{definition}

\begin{definition}[Finite and Infinite Unions]
	For a collection $\{A_i\}$, the union is denoted by $\bigcup_{i} A_i$ and is defined as the set containing all elements that are in at least one of the sets $A_i$.
\end{definition}

\begin{definition}[Partition]
	A collection of non-empty subsets $\{A_i\}$ of a set $A$ is called a partition of $A$ if the following conditions are satisfied:
	\begin{enumerate}
		\item The subsets $A$ are pairwise disjoint, i.e., $A_i \cap A_j = \emptyset$ for all \(i \neq j\).
		\item The union of all subsets \(A_i\) is equal to the set \(A\), i.e., \(\bigcup_{i \in I} A_i = A\).
	\end{enumerate}
	
	A graphical representation of the set $A=\{A_1,A_2,A_3\}$, where $A_j$ are partitions, is shown in \figref{fig:set_partition}.
	\begin{figure}[h]
		\centering
		\includegraphics[width = 0.6\textwidth]{figures/set_partition.pdf}
		\caption{The figure show $A=\{A_1,A_2,A_3\}$ where $A_j$ are partitions.}
		\label{fig:set_partition}
	\end{figure}
\end{definition}


\begin{definition}[Finite and Infinite Intersections]
	For a collection $\{A_i\}$, the intersection is denoted by $\bigcap_{i} A_i$ and is defined as the set containing all elements that are common to all sets $A_i$.
\end{definition}

\begin{definition}[Cartesian Product]
	The Cartesian product of sets $A$ and $B$, denoted by $A \times B$, is defined as the set containing all ordered pairs $(a, b)$, where $a$ is in $A$ and $b$ is in $B$.
\end{definition}

\begin{example}
	Suppose $A= \{a_1,a_2\}$ and $B=\{b_1,b_2,b_3\}$, then
	\begin{equation}
		\begin{split}
			A\times B = \{&(a_1,b_1),(a_1,b_2),(a_1,b_3),\\
			&(a_2,b_1),(a_2,b_2),(a_2,b_3)\}
		\end{split}
	\end{equation}
\end{example}

	
