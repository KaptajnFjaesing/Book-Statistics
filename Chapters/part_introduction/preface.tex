\chapter{Preface}
\label{chp:preface}
Statistics is a mathematical discipline that use probability theory (which in turn require set theory) to extract insights from information (data). Probability theory is a branch of pure mathematics -- probabilistic questions can be posed and solved using axiomatic reasoning, and therefore there is one correct answer to any probability question. Statistical questions can be converted to probability questions by the use of probability models. Given certain assumptions about the mechanism generating the data, statistical questions can be answered using probability theory. This highlights the dual nature of statistics, comprised of two integral parts.
\begin{enumerate}
	\item The first part involves the formulation and evaluation of probabilistic models, a process situated within the realm of the philosophy of science. This phase grapples with the foundational aspects of constructing models that accurately represent the problem at hand.
	\item The second part concerns itself with extracting answers after assuming a specific model. Here, statistics becomes a practical application of probability theory, involving not only theoretical considerations but also numerical analysis in real-world scenarios.
\end{enumerate}
This duality underscores the interdisciplinary nature of statistics, bridging the gap between the conceptual and the applied aspects of probability theory. Although probabilities are well defined (see chapter \ref{chp:probaiblity_theory}), their interpretation is not defined beyond their definition. This ambiguity has given birth to two competing interpretations of probability, leading to two competing branches of statistics; Frequentist and Bayesian Statistics. This book aims to explain how these competing branches of statistics fit together as well as providing a non-exhaustive presentation of some of the methods within both branches. The philosophy of the book is rather straight to the point, but with a lot of examples both big and small. Some of these are anonymized versions of projects from industry. The books is split into three parts; introduction (part \ref{part:introduction}), Frequentist statistics (part \ref{part:frequentist}) and Bayesian statistics (part \ref{part:bayesian}). 


\section{Acknowledgements}
The philosophy of the book is similar to \cite{Sivia2006}, a few exercises from \cite{murphy2023probabilistic} used as examples, the idea of phrasing decision theory as "Robot vs Nature" from \cite{lavalle2006planning} and the review of probability theory is inspired by \cite{chan2021introduction}.