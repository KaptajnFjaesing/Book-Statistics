\chapter{Frequentist Statistics Introduction}
\label{chp:freq}
Frequentist statistics is based on \dfref{def:frequentist_statistics}, which follows the definition of objective probability (\dfref{def:objective_probability}) and the principle of fixed, unknown parameters (\axref{ax:parameter_fixed}). The foundations of Frequentist statistics trace back to seminal works such as those of Neyman and Pearson \citep{Neyman1928OnSR} and Fisher \citep{fisher1925statistical}, who laid the groundwork for much of its methodology. Subsequent developments by Wald \citep{Wald1945Sequential}, Neyman \citep{Neyman1948Consistent}, and Lehmann \citep{lehmann1986testing} further refined its theories and techniques.\newline

In the Frequentist paradigm, it is assumed that Natures decisions can be captured by a model with unknown, fixed parameters $w$. This means everything in \chref{chp:framing_statistics} becomes conditioned on $w$, and the focus becomes estimating $w$ via an estimator $\hat{w}(\tilde{D})$ and subsequently deciding the optimal decision rule
\begin{equation}
	U^*(X=x, \hat{w}(\tilde{D}))
\end{equation}
according to a cost function, as specified in \chref{chp:framing_statistics}.


\begin{example}
	Let $X$ and $S$ be continuous random variables with the relationship~\cite{hastie2001}
	\begin{equation}
		S = f(X,w)+\epsilon
	\end{equation}
	where $\epsilon$ is a random variable representing noise, with $\mathbb{E}[\epsilon]=0$ and $f$ is some model with parameters $w$. Using the quadratic cost function of \dfref{def:quadratic_cost}, \thref{theorem:expectation_cost} yields
	\begin{equation}
		\label{eq:o1}
		\begin{split}
			U^*(X = x,w) &= \mathbb{E}[S|X=x,w,I]\\
			&= f(x,w),
		\end{split}
	\end{equation}
	where it has been used that $f$ does not depend on past data. Hence, if $w$ was known, the Robot could map any observation $X=x$ using the optimal decision rule \EQref{eq:o1}. In reality $w$ is not known and must be estimated.
\end{example}


