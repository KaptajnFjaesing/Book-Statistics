\chapter{Introduction}
\label{chp:introduction}
Statistics is a mathematical discipline that uses probability theory (which, in turn, requires set theory) to extract insights from information (data). Probability theory is a branch of pure mathematics -- probabilistic questions can be posed and solved using axiomatic reasoning, and therefore, there is one correct answer to any probability question. Statistical questions can be converted into probability questions through the use of probability models. Given certain assumptions about the mechanism generating the data, statistical questions can be answered using probability theory. This highlights the dual nature of statistics, which is comprised of two integral parts.
\begin{enumerate}
	\item The first part involves the formulation and evaluation of probabilistic models, a process situated within the realm of the philosophy of science. This phase grapples with the foundational aspects of constructing models that accurately represent the problem at hand.
	\item The second part concerns itself with extracting answers after assuming a specific model. Here, statistics becomes a practical application of probability theory, involving not only theoretical considerations but also numerical analysis in real-world scenarios.
\end{enumerate}
This duality underscores the interdisciplinary nature of statistics, bridging the gap between the conceptual and applied aspects of probability theory. Although probabilities are well defined, their interpretation is not specified beyond their mathematical definition. This ambiguity has given rise to two competing interpretations of probability, leading to two major branches of statistics: Frequentist and Bayesian statistics. This book aims to explain how these competing branches of statistics fit together, as well as to provide a non-exhaustive presentation of some of the methods within both branches.

\newpage
\section{Acknowledgements}
This book has been shaped by many sources of inspiration. A few exercises are adapted from \cite{murphy2023probabilistic}, the idea of presenting decision theory as a contest between ``Robot vs.\ Nature'' is inspired by \cite{lavalle2006planning}, and the overall style has been influenced by works such as \cite{Sivia2006, murphy2023probabilistic, chan2021introduction}.
